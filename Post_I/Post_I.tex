\documentclass[]{article}
\usepackage{lmodern}
\usepackage{amssymb,amsmath}
\usepackage{ifxetex,ifluatex}
\usepackage{fixltx2e} % provides \textsubscript
\ifnum 0\ifxetex 1\fi\ifluatex 1\fi=0 % if pdftex
  \usepackage[T1]{fontenc}
  \usepackage[utf8]{inputenc}
\else % if luatex or xelatex
  \ifxetex
    \usepackage{mathspec}
  \else
    \usepackage{fontspec}
  \fi
  \defaultfontfeatures{Ligatures=TeX,Scale=MatchLowercase}
\fi
% use upquote if available, for straight quotes in verbatim environments
\IfFileExists{upquote.sty}{\usepackage{upquote}}{}
% use microtype if available
\IfFileExists{microtype.sty}{%
\usepackage{microtype}
\UseMicrotypeSet[protrusion]{basicmath} % disable protrusion for tt fonts
}{}
\usepackage[margin=1in]{geometry}
\usepackage{hyperref}
\hypersetup{unicode=true,
            pdftitle={Post I: Driving mammals into darkness},
            pdfborder={0 0 0},
            breaklinks=true}
\urlstyle{same}  % don't use monospace font for urls
\usepackage{graphicx,grffile}
\makeatletter
\def\maxwidth{\ifdim\Gin@nat@width>\linewidth\linewidth\else\Gin@nat@width\fi}
\def\maxheight{\ifdim\Gin@nat@height>\textheight\textheight\else\Gin@nat@height\fi}
\makeatother
% Scale images if necessary, so that they will not overflow the page
% margins by default, and it is still possible to overwrite the defaults
% using explicit options in \includegraphics[width, height, ...]{}
\setkeys{Gin}{width=\maxwidth,height=\maxheight,keepaspectratio}
\IfFileExists{parskip.sty}{%
\usepackage{parskip}
}{% else
\setlength{\parindent}{0pt}
\setlength{\parskip}{6pt plus 2pt minus 1pt}
}
\setlength{\emergencystretch}{3em}  % prevent overfull lines
\providecommand{\tightlist}{%
  \setlength{\itemsep}{0pt}\setlength{\parskip}{0pt}}
\setcounter{secnumdepth}{0}
% Redefines (sub)paragraphs to behave more like sections
\ifx\paragraph\undefined\else
\let\oldparagraph\paragraph
\renewcommand{\paragraph}[1]{\oldparagraph{#1}\mbox{}}
\fi
\ifx\subparagraph\undefined\else
\let\oldsubparagraph\subparagraph
\renewcommand{\subparagraph}[1]{\oldsubparagraph{#1}\mbox{}}
\fi

%%% Use protect on footnotes to avoid problems with footnotes in titles
\let\rmarkdownfootnote\footnote%
\def\footnote{\protect\rmarkdownfootnote}

%%% Change title format to be more compact
\usepackage{titling}

% Create subtitle command for use in maketitle
\newcommand{\subtitle}[1]{
  \posttitle{
    \begin{center}\large#1\end{center}
    }
}

\setlength{\droptitle}{-2em}

  \title{Post I: Driving mammals into darkness}
    \pretitle{\vspace{\droptitle}\centering\huge}
  \posttitle{\par}
    \author{}
    \preauthor{}\postauthor{}
      \predate{\centering\large\emph}
  \postdate{\par}
    \date{08/08/2018}


\begin{document}
\maketitle

\begin{center}\rule{0.5\linewidth}{\linethickness}\end{center}

Atypical for a zoologist, I'm terrified of most animals. Growing up in
the wildlife paradise of Stoke-on-Trent, my mother's response was a
fairly standard ``they're more scared of you than you are of them''.
Maybe that's not true of my childhood interactions with Canadian Geese -
but new research published in the journal
\href{http://science.sciencemag.org/content/360/6394/1232}{\emph{Science}}
highlights widespread changes in the behaviour of mammals, which the
authors attribute to fear of humans. We're driving our fellow mammals
into the night: increasing nocturnal activity in the presence of human
activity, on average, by a factor of 1.36\protect\hyperlink{f1}{1}. This
means, for example, that if a species naturally split its activity 50\%
day and 50\% night, around humans the night time activity would increase
to 68\%.

The black bear, \emph{Ursus americanus}, is amongst large predetors
showing increased night time activity in response to people. Image:
Cephas/Wikimedia Commons, CC BY-SA 3.0

What is it that we're doing to push our Class-mates under the cover of
darkness? The authors pulled together 76 studies, from across the globe,
measuring activity patterns in relation to anthropogenic stressors
including vehicles, agriculture, mining, hunting, building works, and
even hiking. Significant increases to nocturnality were found in
response to \emph{every single form} of human presence that they tested.
Surprisingly, the diel pattern shifts caused by non-lethal activities
like hiking and other recreational activities were statistically
indistinguishable from those caused by lethal activities such as hunting
- which the authors say suggests ``that animals perceive and respond to
human threats even when they pose no direct
risk''\protect\hyperlink{f1}{1}.

These findings were consistent across habitats, continents, normal
activity periods, diet types, and body size classes. Even apex predators
like the lion (\emph{Panthera leo}) and the black bear (\emph{Ursus
americanus}), which have recent evolutionary histories largely devoid of
predation risk, appear to be responding to human activity by increased
nocturnality.

These findings sound scary: humans, acting as a diurnal apex ``super
predator'', not content with constricting animal spatial distributions
through massive habitat destruction and defaunation are also affecting
their temporal distribution too.

The cheetah, \emph{Acinonyx jubatus}, may already owe its diel pattern
to temporal niche partitioning with dominant lions - it too shows
increasing nocturnality in areas of human disturbance. Image: Amit
Patel, CC-0

Some optimists may see this as a positive thing - evidence of adaptation
in action, and there is some evidence that where people lethally
threaten animals a shift towards nocturnality can increase survival
probability\protect\hyperlink{f2}{2}.

Indeed, \emph{temporal niche partitioning} (to give it it's scientific
name) is an intrinsic feature of Earth's ecosystems. Cheetahs
(\emph{Acinonyx jubatus}) and lions, for example, are large predators
with extensive dietary and spatial overlap. The larger lion is known to
kill cheetahs, prevent them from accessing prey-dense areas, and steal
their food - but they manage to coexist. Some suggest that this is a
result of temporal niche partitioning, the nocturnal lions forging the
(largely) diurnal activity pattern of the cheetah over evolutionary
time, but the evidence supporting this is far from
clear\protect\hyperlink{f3}{3}.

Considering their history, it seems likely that diurnal mammals will
possess traits and sensory adaptations that work optimally in daylight.
Driven into the night, amongst other effects these animals may no longer
be able to hunt or forage as efficiently, their social behaviour may be
disrupted, they may be poorer at navigation. All of these could set off
a domino-effect cascade throughout the trophic levels of an ecosystem:
increasing nocturnality in prey animals puts them at risk from nocturnal
predators; increasing nocturnality in predators may push prey animals
towards diurnality and increase human-interactions themselves.

Understanding how ecological communities will change in response to
human activity, for better or worse, requires further study. I'd be
especially interested to see whether similar diel shifts are found
across non-mammalian animals - though Canadian Geese continue to
terrorise me, personally!

-- Adam

\begin{center}\rule{0.5\linewidth}{\linethickness}\end{center}

\emph{References:}\\
1 Gaynor, K. M., C. E. Hojnowski, N. H. Carter, \& J. S. Brashares,
2018. The influence of human disturbance on wildlife nocturnality.
\emph{Science} \textbf{360}, 1232-1235. \protect\hyperlink{a1}{↩︎}\\
2 Murray, M. H., \& C. C. St.~Clair, 2015. Individual flexibility in
nocturnal activity reduces risk of road mortality for an urban
carnivore. \emph{Behavioral Ecology} \textbf{26}, 1520-1527.
\protect\hyperlink{a2}{↩︎}\\
3 Cozzi, G., F. Broekhuis, J. W. McNutt, L. A. Turnbull, D. W.
Macdonald, \& B. Schmid, 2012. Fear of dark or dinner by moonlight?
Reduced temporal partitioning among Africa's large carnivores.
\emph{Ecology} \textbf{93}, 2590-2599. \protect\hyperlink{a3}{↩︎}


\end{document}
